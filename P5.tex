\documentclass[conference]{IEEEtran}

\usepackage{cite}
\usepackage{amsmath,amssymb,amsfonts}
\usepackage{algorithmic}
\usepackage{graphicx}
\usepackage{textcomp}
\usepackage{xcolor}
\def\BibTeX{{\rm B\kern-.05em{\sc i\kern-.025em b}\kern-.08em
    T\kern-.1667em\lower.7ex\hbox{E}\kern-.125emX}}
\begin{document}

\title{Usability Test Plan \\
\large Social Media App for Pickleball Players}

\author{
\IEEEauthorblockN{
    Rylan Melcher\\
    Sean Stara\\
    Ethan Stenger\\
    Narayana Pujara
}

\IEEEauthorblockA{
    School of Engineering and School of Computing \\
    University of Nebraska–Lincoln \\
    Lincoln, NE
}
}
\maketitle

\section{Usability Testing Overview}

This usability test aims to evaluate the clarity, efficiency, and overall user experience of our prototype social media application for pickleball players. The study focuses on how well users can navigate the interface, complete essential actions, and understand key features such as player discovery, event organization, matchmaking, content posting, and skill tracking. A group of 5--7 active pickleball players who regularly use mobile apps and social media will participate in short, task-based sessions lasting 25--35 minutes. During these sessions, participants will complete realistic workflows while thinking aloud, allowing us to observe their decision-making, identify points of confusion, and capture both behavioral and verbal feedback. Quantitative data such as task completion, errors, and time on task will be collected alongside qualitative insights from observations and a brief post-test interview. The results of this evaluation will guide refinements to the prototype and ensure the final design effectively supports pickleball players seeking community, competition, and connection through the platform.



\section{Hypotheses}
Users will be able to create an account and set up their profile within 5 minutes without assistance. They will also be able to successfully find and create a match within 10 minutes of opening the website. This is in order to ensure our product is easy to use as well as efficient to maintain. This 'within' subject approach will be more effective than it's counterpart in our situation because we don't have many dependent variables to change, but are looking to improve the user experience through ease of use and time of tasks.
\section{Participants and Their Descriptions} %Methods/Apparatus


\section{Experiment Methods and Design}
\label{Sec:Results}
This section presents your results, because it has a label you can refer to it in other sections by saying something you can say something like: this is further discussed in Section \ref{Sec:Results}. In general, you typically present means and standard deviations followed by the full statistical analysis. You are to discuss which results are significant or not (unlikely that any will be). What may cause a result to be significant or not, etc. Graphs, charts and tables are usually good to include, but only if they provide important details not available in a table or text. A statistical result with p <= 0.05 is consider significant. When formatting data this site may be helpful for APA style: \begin{verbatim}
https://owl.purdue.edu/owl/
research_and_citation/apa_style/
apa_formatting_and_style_guide/
apa_tables_and_figures.html
\end{verbatim}

\section{Data Analysis}
This section is to discuss your results and what exactly they mean. What are the implications regarding interface design based upon the results? How do the results generalize or not generalize, etc? This section also needs to discuss what you can change about your system based on the evaluation feedback.


\end{document}
