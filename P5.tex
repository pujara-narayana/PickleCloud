\documentclass[conference]{IEEEtran}

\usepackage{cite}
\usepackage{amsmath,amssymb,amsfonts}
\usepackage{algorithmic}
\usepackage{graphicx}
\usepackage{textcomp}
\usepackage{xcolor}
\def\BibTeX{{\rm B\kern-.05em{\sc i\kern-.025em b}\kern-.08em
    T\kern-.1667em\lower.7ex\hbox{E}\kern-.125emX}}
\begin{document}

\title{Usability Test Plan \\
\large Social Media App for Pickleball Players}

\author{
\IEEEauthorblockN{
    Rylan Melcher\\
    Sean Stara\\
    Ethan Stenger\\
    Narayana Pujara
}

\IEEEauthorblockA{
    School of Engineering and School of Computing \\
    University of Nebraska–Lincoln \\
    Lincoln, NE
}
}
\maketitle

\section{Usability Testing Overview}

This usability test aims to evaluate the clarity, efficiency, and overall user experience of our prototype social media application for pickleball players. The study focuses on how well users can navigate the interface, complete essential actions, and understand key features such as player discovery, event organization, matchmaking, content posting, and skill tracking. A group of 5--7 active pickleball players who regularly use mobile apps and social media will participate in short, task-based sessions lasting 25--35 minutes. During these sessions, participants will complete realistic workflows while thinking aloud, allowing us to observe their decision-making, identify points of confusion, and capture both behavioral and verbal feedback. Quantitative data such as task completion, errors, and time on task will be collected alongside qualitative insights from observations and a brief post-test interview. The results of this evaluation will guide refinements to the prototype and ensure the final design effectively supports pickleball players seeking community, competition, and connection through the platform.



\section{Hypotheses}
Users will be able to create an account and set up their profile within 5 minutes without assistance. They will also be able to successfully find and create a match within 10 minutes of opening the website. This is in order to ensure our product is easy to use as well as efficient to maintain. This 'within' subject approach will be more effective than it's counterpart in our situation because we don't have many dependent variables to change, but are looking to improve the user experience through ease of use and time of tasks.
\section{Participants and Their Descriptions} %Methods/Apparatus


\section{Experiment Methods and Design}
    The usability testing sessions for PickleCloud will begin with a brief introduction. Participants will be welcomed and informed of the purpose of the study, to evaluate the intuitiveness and efficiency of the app. No training will be provided, as the goal is to observe natural user behavior. 
    
    Each participant will complete three tasks with increasing complexity. Task 1 is to "find a match with an evenly skilled opponent based on ELO ratings." Task 2 is to "view a map of local courts within the app, including real-time availability, court features, and user ratings." Task 3 is to "chat with opponents before a match to coordinate the time, and then use the score tracker for live score tracking once the game begins". These tasks will be read aloud by the facilitator, who will remain silent unless the participant is stuck for more than two minutes.
    
    The session will follow a structured flow: introduction (2-5 minutes), task execution (20–30 minutes), post-test survey (5 minutes), and a debrief discussion (5 minutes). After completing the tasks, the participants will complete a post-test survey that includes task-specific ease and satisfaction ratings, and open-ended questions about their experience. Data will be collected through stopwatch timing, satisfaction scores, and written observer notes. Together, these methods will give a clear picture of how people actually experience PickleCloud and highlight areas where the app can be improved.

\section{Data Analysis}
    For Hypotheses 1 and 2 (account creatation and profile customization), task completion will be counted on wether you did or didnt accomplish it. A lower amount of completions in our prototype will support this.
    
    For Hypotheses 3, task completion will be averaged with sub goals to reach so we can know where users may get tripped up on. The prototype showing a higher average with support our claims.


\end{document}
