\documentclass[conference]{IEEEtran}

\usepackage{cite}
\usepackage{amsmath,amssymb,amsfonts}
\usepackage{algorithmic}
\usepackage{graphicx}
\usepackage{textcomp}
\usepackage{xcolor}
\def\BibTeX{{\rm B\kern-.05em{\sc i\kern-.025em b}\kern-.08em
    T\kern-.1667em\lower.7ex\hbox{E}\kern-.125emX}}
\begin{document}

\title{Usability Test Plan \\
\large Social Media App for Pickleball Players}

\author{
\IEEEauthorblockN{
    Rylan Melcher\\
    Sean Stara\\
    Ethan Stenger\\
    Narayana Pujara
}

\IEEEauthorblockA{
    School of Engineering and School of Computing \\
    University of Nebraska–Lincoln \\
    Lincoln, NE
}
}
\maketitle

\section{Usability Testing Overview}

This usability test aims to evaluate the clarity, efficiency, and overall user experience of our prototype social media application for pickleball players. The study focuses on how well users can navigate the interface, complete essential actions, and understand key features such as player discovery, event organization, matchmaking, content posting, and skill tracking. A group of 5--7 active pickleball players who regularly use mobile apps and social media will participate in short, task-based sessions lasting 25--35 minutes. During these sessions, participants will complete realistic workflows while thinking aloud, allowing us to observe their decision-making, identify points of confusion, and capture both behavioral and verbal feedback. Quantitative data such as task completion, errors, and time on task will be collected alongside qualitative insights from observations and a brief post-test interview. The results of this evaluation will guide refinements to the prototype and ensure the final design effectively supports pickleball players seeking community, competition, and connection through the platform.



\section{Hypotheses}
\subsection{Hypothesis 1: Task Efficiency}

At least 90\% of participants will complete the core match-finding workflow (viewing local courts, finding a match with an evenly skilled opponent, and starting chat to confirm details) in under 10 minutes on their first attempt using Pickle Cloud, which will be faster than the time they currently spend coordinating pickup games through traditional methods.
\\
\subsubsection*{Explanation}
This hypothesis tests Pickle Cloud's efficiency and clarity, which combines "player discovery" with "matchmaking." We are focused on "how well users can complete essential actions." If this is confusing or time-consuming, we risk users abandoning our app. We hope to learn if the navigation and feature integration are as intuitive as intended.
\\
\subsubsection*{Design}
We will be using "within subjects design" because it controls for individual differences in technical proficiency and pickleball experience. Each participant will complete the full match-finding workflow using Pickle Cloud.

\subsubsection*{Independent Variable}
The user interface and user experience design of the Pickle Cloud prototype's court-finding and matchmaking features.
\\
\subsubsection*{Dependent Variable}
Time to complete match arrangement workflow (in minutes).

\subsection{Hypothesis 2: Interface Clarity and User Satisfaction}

Participants will find the Pickle Cloud to be intuitive and highly satisfying to use as they will easily be able to complete all three core tasks (1. view and select a local court, 2. find a match with an evenly skilled opponent, 3. start a chat to confirm match details) without assistance or critical errors on their first attempt and give an average of 85\% satisfaction score (out of 100\%).
\\
\subsubsection*{Explanation}
This hypothesis evaluates the "overall user experience" and the app's potential to foster "community and connection," as stated in our overview. While Hypothesis 1 measures the speed of a task, this measures subjective satisfaction and perceived ease of use. This will confirm that users find the app integrated and easy to learn, which is important for a social platform app like ours.
\\
\subsubsection*{Design}
We will be using "within subjects design" because we are evaluating the inherent usability of a single prototype interface, not comparing design alternatives and each task tests a distinct feature area (courts, matchmaking, chat), minimizing learning effect interference. All participants will attempt the same three core tasks in sequence.
\\
\subsubsection*{Independent Variable}
The overall user experience and UI design of the Pickle Cloud prototype.
\\
\subsubsection*{Dependent Variables}
Qualitative comments and satisfaction rate (out of 100\%).
\section{Participants and Their Descriptions} %Methods/Apparatus


\section{Experiment Methods and Design}
    The usability testing sessions for PickleCloud will begin with a brief introduction. Participants will be welcomed and informed of the purpose of the study, to evaluate the intuitiveness and efficiency of the app. No training will be provided, as the goal is to observe natural user behavior. 
    
    Each participant will complete three tasks with increasing complexity. Task 1 is to "find a match with an evenly skilled opponent based on ELO ratings." Task 2 is to "view a map of local courts within the app, including real-time availability, court features, and user ratings." Task 3 is to "chat with opponents before a match to coordinate the time, and then use the score tracker for live score tracking once the game begins". These tasks will be read aloud by the facilitator, who will remain silent unless the participant is stuck for more than two minutes.
    
    The session will follow a structured flow: introduction (2-5 minutes), task execution (20–30 minutes), post-test survey (5 minutes), and a debrief discussion (5 minutes). After completing the tasks, the participants will complete a post-test survey that includes task-specific ease and satisfaction ratings, and open-ended questions about their experience. Data will be collected through stopwatch timing, satisfaction scores, and written observer notes. Together, these methods will give a clear picture of how people actually experience PickleCloud and highlight areas where the app can be improved.

\section{Data Analysis}

    For Hypothesis 1, we want to see if participants can complete the core match-finding workflow on Pickle Cloud efficiently. We'll record whether each participant completes the workflow (\emph{completed} or \emph{not completed}) and calculate frequency counts and percentages. If 90\% or more of participants complete the workflow, it will support the idea that Pickle Cloud improves task efficiency.

    For Hypothesis 2, the focus is on how clear and satisfying the interface is. Each of the three main tasks (view/select a court, find a match, start a chat) will be tracked as completed or not. Participants will also rate their satisfaction on a 0–100\% scale. We'll report task completion percentages and the average satisfaction score. High completion rates and an average satisfaction of 85\% or more would indicate that users find the interface intuitive and enjoyable to use.

    For Hypothesis 3, we aim to pinpoint where participants might struggle in the workflow. Each task will be broken into sub-goals, and we'll calculate an average completion rate across these sub-goals for each participant. The mean and standard deviation of these averages will be reported. A higher overall average would support the hypothesis that the interface effectively guides users through all steps.

\end{document}