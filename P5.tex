\documentclass[conference]{IEEEtran}

\usepackage{cite}
\usepackage{amsmath,amssymb,amsfonts}
\usepackage{algorithmic}
\usepackage{graphicx}
\usepackage{textcomp}
\usepackage{xcolor}
\def\BibTeX{{\rm B\kern-.05em{\sc i\kern-.025em b}\kern-.08em
    T\kern-.1667em\lower.7ex\hbox{E}\kern-.125emX}}
\begin{document}

\title{Usability Test Plan \\
\large Social Media App for Pickleball Players}

\author{
\IEEEauthorblockN{Rylan Melcher}
\IEEEauthorblockA{
School of Engineering, University of Nebraska–Lincoln \\
Lincoln, NE \\ 
}

\and

\IEEEauthorblockN{Sean Stara}
\IEEEauthorblockA{
School of Computing, University of Nebraska–Lincoln \\
Lincoln, NE \\
}

\and

\IEEEauthorblockN{Ethan Stenger}
\IEEEauthorblockA{
School of Computing, University of Nebraska–Lincoln \\
Lincoln, NE \\
}

\and

\IEEEauthorblockN{Narayana Pujara}
\IEEEauthorblockA{
School of Computing, University of Nebraska–Lincoln \\
Lincoln, NE \\
}
}


\maketitle


\section{Usability Testing Overview}

This usability test aims to evaluate the clarity, efficiency, and overall user experience of our prototype social media application for pickleball players. The study focuses on how well users can navigate the interface, complete essential actions, and understand key features such as player discovery, event organization, matchmaking, content posting, and skill tracking. A group of 5--7 active pickleball players who regularly use mobile apps and social media will participate in short, task-based sessions lasting 25--35 minutes. During these sessions, participants will complete realistic workflows while thinking aloud, allowing us to observe their decision-making, identify points of confusion, and capture both behavioral and verbal feedback. Quantitative data such as task completion, errors, and time on task will be collected alongside qualitative insights from observations and a brief post-test interview. The results of this evaluation will guide refinements to the prototype and ensure the final design effectively supports pickleball players seeking community, competition, and connection through the platform.



\section{System Description}
Describes the interface design, important requirements and design specifics.

\section{Methods} %Methods/Apparatus

\subsection{Hypotheses}
Your hypotheses explain what you expect to learn or confirm from your user testing. You should include at least two, and at least one must be quantitative (objective).

\subsection{Participants}
Details how many participants, who they were (i.e. UNL graduate students, etc.), any important screening criteria and demographics (such as age).

\subsection{Experimental Design}
Details how you set up the design including: is it a within or between subjects design, the interface conditions, dependent and independent variables, subjective and objective metrics, number of trials, presentation of trials, how and when data was collected, etc.

\subsection{Data Analysis}
Explain how you will analyze the data you collect to answer your research questions.

\section{Results}
\label{Sec:Results}
This section presents your results, because it has a label you can refer to it in other sections by saying something you can say something like: this is further discussed in Section \ref{Sec:Results}. In general, you typically present means and standard deviations followed by the full statistical analysis. You are to discuss which results are significant or not (unlikely that any will be). What may cause a result to be significant or not, etc. Graphs, charts and tables are usually good to include, but only if they provide important details not available in a table or text. A statistical result with p <= 0.05 is consider significant. When formatting data this site may be helpful for APA style: \begin{verbatim}
https://owl.purdue.edu/owl/
research_and_citation/apa_style/
apa_formatting_and_style_guide/
apa_tables_and_figures.html
\end{verbatim}

\section{Discussion}
This section is to discuss your results and what exactly they mean. What are the implications regarding interface design based upon the results? How do the results generalize or not generalize, etc? This section also needs to discuss what you can change about your system based on the evaluation feedback.

\section{Conclusions}


%The IEEEtran class file is used to format your paper and style the text. All margins, column widths, line spaces, and text fonts are prescribed; please do not alter them. You may note peculiarities. For example, the head margin measures proportionately more than is customary. This measurement and others are deliberate, using specifications that anticipate your paper as one part of the entire proceedings, and not as an independent document. Please do not revise any of the current designations.

\section{Assorted \LaTeX notes}
\subsection{Abbreviations and Acronyms}\label{AA}
Define abbreviations and acronyms the first time they are used in the text, even after they have been defined in the abstract.  Do not use abbreviations in the title or heads unless they are unavoidable.

\subsection{Units}
\begin{itemize}
\item Use either SI (MKS) or CGS as primary units. (SI units are encouraged.) English units may be used as secondary units (in parentheses). An exception would be the use of English units as identifiers in trade, such as ``3.5-inch disk drive''.
\item Avoid combining SI and CGS units, such as current in amperes and magnetic field in oersteds. This often leads to confusion because equations do not balance dimensionally. If you must use mixed units, clearly state the units for each quantity that you use in an equation.
\item Do not mix complete spellings and abbreviations of units: ``Wb/m\textsuperscript{2}'' or ``webers per square meter'', not ``webers/m\textsuperscript{2}''. Spell out units when they appear in text: ``. . . a few henries'', not ``. . . a few H''.
\item Use a zero before decimal points: ``0.25'', not ``.25''. Use ``cm\textsuperscript{3}'', not ``cc''.)
\end{itemize}

\subsection{Equations}
Number equations consecutively. To make your 
equations more compact, you may use the solidus (~/~), the exp function, or 
appropriate exponents. Italicize Roman symbols for quantities and variables, 
but not Greek symbols. Use a long dash rather than a hyphen for a minus 
sign. Punctuate equations with commas or periods when they are part of a 
sentence, as in:
\begin{equation}
a+b=\gamma\label{eq}
\end{equation}

Be sure that the symbols in your equation have been defined before or immediately following the equation. Use ``\eqref{eq}'', not ``Eq.~\eqref{eq}'' or ``equation \eqref{eq}'', except at the beginning of a sentence: ``Equation \eqref{eq} is . . .''


\subsection{Some Common Mistakes}\label{SCM}
\begin{itemize}
\item The word ``data'' is plural, not singular.
\item The subscript for the permeability of vacuum $\mu_{0}$, and other common scientific constants, is zero with subscript formatting, not a lowercase letter ``o''.
\item In American English, commas, semicolons, periods, question and exclamation marks are located within quotation marks only when a complete thought or name is cited, such as a title or full quotation. When quotation marks are used, instead of a bold or italic typeface, to highlight a word or phrase, punctuation should appear outside of the quotation marks. A parenthetical phrase or statement at the end of a sentence is punctuated outside of the closing parenthesis (like this). (A parenthetical sentence is punctuated within the parentheses.)
\item A graph within a graph is an ``inset'', not an ``insert''. The word alternatively is preferred to the word ``alternately'' (unless you really mean something that alternates).
\item Do not use the word ``essentially'' to mean ``approximately'' or ``effectively''.
\item In your paper title, if the words ``that uses'' can accurately replace the word ``using'', capitalize the ``u''; if not, keep using lower-cased.
\item Be aware of the different meanings of the homophones ``affect'' and ``effect'', ``complement'' and ``compliment'', ``discreet'' and ``discrete'', ``principal'' and ``principle''.
\item Do not confuse ``imply'' and ``infer''.
\item The prefix ``non'' is not a word; it should be joined to the word it modifies, usually without a hyphen.
\item There is no period after the ``et'' in the Latin abbreviation ``et al.''.
\item The abbreviation ``i.e.'' means ``that is'', and the abbreviation ``e.g.'' means ``for example''.
\end{itemize}
An excellent style manual for science writers is \cite{b7}.


\subsection{Figures and Tables}
\paragraph{Positioning Figures and Tables} Place figures and tables at the top and 
bottom of columns. Avoid placing them in the middle of columns. Large 
figures and tables may span across both columns. \textbf{Figure captions should be 
below the figures; table heads should appear above the tables}. Insert 
figures and tables after they are cited in the text. Use the abbreviation 
``Fig.~\ref{fig}'', even at the beginning of a sentence.

\begin{table}[htbp]
\caption{Table Type Styles}
\begin{center}
\begin{tabular}{|c|c|c|c|}
\hline
\textbf{Table}&\multicolumn{3}{|c|}{\textbf{Table Column Head}} \\
\cline{2-4} 
\textbf{Head} & \textbf{\textit{Table column subhead}}& \textbf{\textit{Subhead}}& \textbf{\textit{Subhead}} \\
\hline
copy& More table copy$^{\mathrm{a}}$& &  \\
\hline
\multicolumn{4}{l}{$^{\mathrm{a}}$Sample of a Table footnote.}
\end{tabular}
\label{tab1}
\end{center}
\end{table}

\begin{figure}[htbp]
\centerline{\includegraphics{fig1.png}}
\caption{Example of a figure caption.}
\label{fig}
\end{figure}

Figure Labels: Use 8 point Times New Roman for Figure labels. Use words 
rather than symbols or abbreviations when writing Figure axis labels to 
avoid confusing the reader. As an example, write the quantity 
``Magnetization'', or ``Magnetization, M'', not just ``M''. If including 
units in the label, present them within parentheses. Do not label axes only 
with units. In the example, write ``Magnetization (A/m)'' or ``Magnetization 
\{A[m(1)]\}'', not just ``A/m''. Do not label axes with a ratio of 
quantities and units. For example, write ``Temperature (K)'', not 
``Temperature/K''.

\end{document}
